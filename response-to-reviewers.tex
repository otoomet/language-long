% -*- mode: LaTeX; mode: flyspell; -*-
% mode: Tex-Pdf -*-
\documentclass{article}
\usepackage{amsmath}
\usepackage[T1]{fontenc}
\usepackage[bookmarks=TRUE,
            colorlinks,
            pdfpagemode=none,
            pdfstartview=FitH,
            citecolor=black,
            filecolor=black,
            linkcolor=blue,
            urlcolor=black,
            ]{hyperref}
\usepackage[top=15mm, bottom=15mm, left=10mm, textwidth=130mm, marginparwidth=70mm]{geometry}
\usepackage{graphicx}
\usepackage{icomma}
\usepackage[utf8]{inputenc}
\usepackage{natbib}
\usepackage{setspace}
\usepackage[dvipsnames]{xcolor}
\usepackage{xspace}
\usepackage{csquotes}
\usepackage[layout=margin,author=,status=draft]{fixme}

\fxsetup{theme=color}
\newcommand{\skb}[1]{
	\begin{quotation}
		\color{Blue} SKB: #1	
	\end{quotation}
}
\newcommand{\debug}[1]{\marginpar{\small\raggedright\begin{spacing}{1}
      \color{BrickRed} #1
\end{spacing}
}}
\newenvironment{response}
{\slshape}{}

\begin{document}
	\section*{General remarks} % which are too long for the margin... .
\begin{itemize}
	\item A rather general remark, I added the occupation variable to the estimations, but did not redo the graphs or tables. The results should not differ much. It will show up in the tables with all coefficients.
\item SKB: The column separating lines in Table 4 have some \enquote{holes}. We might have to switch from the dcolumn package to the siunitx package to fix these issues. I did not manage yet fix these issues. The referees did not complain, but it still looks ugly or strange. 
\end{itemize}
\section*{Response to the referees}

\section{Comments by the editor}

\begin{enumerate}
\item In addition, please elaborate further the comparison of your
  results with respect to the previous literature about the role of
  language in Baltic, CEE, and former Soviet Union countries. 
\item 
In revising your paper, please make an effort to keep within the
journal's word-length norm of 7500 words (including references, notes
and appendices, but not figures and tables) and restrict the number of
references to 50 items.

\debug{nothing else but 7500 words/50 items.}
\item When submitting your revised manuscript, you will be able to respond to the comments made by the referee(s) in the space provided.  You can use this space to document any changes you make to the original manuscript.  In order to expedite the processing of the revised manuscript, please be as specific as possible in your response to the referee(s).
\end{enumerate}

\begin{response}
  At a reviewers request, we have added the full regression tables in
  Appendix.  We agree these figures are valuable, in particular in the
  light of the striving toward the replicable research in the recend
  years, but we think they fit better in an online supplement.
\end{response}

\section{Referee 1}

\subsection{Main comments:}


\begin{enumerate}
\item The paper aims to contribute to an already large
  literature. The recent literature in this field focus on the
  identification of causal effects by using instrumental variable
  approaches (see, e.g., Isphording et al., 2014; Yao and van Ours,
  2015). However, I like the fact that the authors a very clear that
  they are not able to identify causal effects. This is, of course, a
  shortcoming and is limiting the overall contribution of the
  paper. Given the specific focus of International Journal of Manpower
  and the interests of its readership, I find that the paper is
  nevertheless providing a small, but still interesting
  contribution. By analysing data from Estonia and providing results
  on natives’ language skills the paper distinguished itself form
  large parts of the literature and adds some new results to our
  knowledge on the labour market effects of language skills.

  \begin{response}
    No response needed here.
  \end{response}
\item 
  My main concern is with some of their wage results. I find it
  puzzling that they don’t find a wage premium for Russian men who
  have Estonian language skills as they do find a sizeable significant
  effect of about 11 to 17 percent for Russian women and a significant
  positive effect for Estonians who speak Russian. Furthermore, they
  find that Estonian language skills do reduce the probability of
  being unemployed for Russians. Why is there a reduction in the
  probability of being unemployed, but no wage effect? Can we trust
  this result? Looking at the descriptive statistics (see Table 1) the
  wage for Russian men who speak Estonian seems to be higher compared
  to Russian men without the ability to speak Estonian (489.96 vs
  401.59). The authors should either provide more statistics on this
  result (if possible) or a more detailed discussion on potential
  reasons that lead to this finding.

  \begin{response}
    This is a good point.  We have added some more analysis (see
    Section 7), and it appears the story is somewhat more nuanced.
  \end{response}
\item
  \debug{OT: I'll do a rewrite after changes are made, hope it's
    sufficient.} 
  The paper needs to be checked for spelling and grammar
  errors. There are a number of typos and errors, which should be
  corrected in a revision. This would normally be a minor comment, but
  here the authors should take this comment very serious as there are
  so many mistakes.
\end{enumerate}

\subsection{Minor comments:}

\begin{enumerate}
\item- The description of the estimation models in Chapter 4 should
  be more detailed. Please describe more specifically what exactly you
  are estimating. In particular the description of the second model
  estimating the time trend is very short.

  \begin{response}
    We have improved and simplified the description of the model and related variables.
  \end{response}
  \fxwarning{ot: should you cluster if you do random effects?}
\item - The tables and figures should have notes that explain in more
  detail what is shown in the respective table and figure. For
  instance, the table notes should contain the included control
  variables, the sample period and the age restriction applied.

  \begin{response}
    We have improved the captions and notes.
  \end{response}

\item page 4: You write that you are only analysing individuals who are
working or actively looking for a job, but in line 35 on the same page
you write that inactivity is one of the included labour market
outcomes. Please clarify this and be clear in your description of what
is included in your sample.

\begin{response}
  We have removed the erroneous mentioning of inactivity 
\end{response}
\item  The sample size for different specification of the same analysis
  should be the equal. Otherwise it is difficult to conclude whether the
  changes in the coefficients are coming from the addition of more
  controls or form the change in the size of the estimation sample.

  \begin{response}
    Now we report the models with identical results.  The original
    (very similar)
    results are still available in Appendix.
  \end{response}

\item In the paper, you are only showing the results of the variables of
  interest. I suggest to provide tables including all coefficients (not
  the year dummies) in the Appendix. It might be interesting to see how
  the coefficients of the control variables change across the different
  models.

  \begin{response}
    We have added the full wage results in Appendix.  We think these
    tables would be more suitable for an online supplement.
    \fxerror{ot: if we do wage, we should also do unemployment}
  \end{response}
\item 
- All tables: Be consistent in the description of your tables. The
table notes and the labels for e.g., "observations", are inconsistent
across the different tables. 
\debug{SKB: Fixed. Changed obs to Observations in tables}
\item
  \debug{ot: this was because of both 'B. Chiswick' and
    'B. R. Chiswick' in the bibtex file, and bibtex wanted to
    distinguish between these two guys.}
 Page 1: To cite Chiswick you are using different versions
(B. Chiswick and B. R. Chiswick).

\begin{response}
  We have corrected the different references.
\end{response}

\item 
- Page 2, line 58: Not the unemployment rate, but the probability of
being unemployed is reduced.
\debug{SKB: Got fixed by Svetlana I think.}
\item 
- Page 3, lines 46-50: You are using a comma as the decimal separator, but in all your tables you are using a point. Please use a point everywhere. 
\debug{SKB: Fixed by someone.}
\end{enumerate}



\section{Referee: 2}


\subsection{Additional Questions:}

\subsubsection*{4. Results}

\begin{enumerate}
\item For this papers findings section in the abstract they imply that
  for all men fluency in Estonian is related to 5\% lower unemployment
  and no income premium.  From Table 2 and 3 it appears that this
  should be for only Russian males. English skills are associated with
  a large income premium and no unemployment. From Table 2 and 3 this
  appears to be more accurately just Russians, as the Estonians appear
  to have a significant negative correlation with unemployment.  This
  is supported by what is said later in the Discussion and Concluding
  Remarks section in what they state is their most intriguing
  observation.  Claiming this to be the most intriguing is interesting
  as it seems to support some what previous literature has stated and
  one of the extensions that this paper is making the inclusion of
  women and namely how their effects differ from males and previous
  literature. 
\item 
\debug{OT: seems like I have to address this.}
In the Discussion and Concluding Remarks section they state that
  “The lower end labour market, associated with frequent unemployment,
  is better integrated and those who are fluent in the language may
  expect both less unemployment and higher wage. However, men at
  better paid and more stable jobs gain little form Estonian” (the
  language I presume) This conclusion felt surprising when reading the
  paper as it did not feel fully fleshed out in the results section
  how this was observed. Tying this to the results would be helpful
  and make it clearer exactly where these results come from.
\item In Table 3 I am perplexed how the number of observations increases as more restrictions are added to the model. How could running the same model but including additional workplace controls increase the observations from 7978 to 8233 for men and 9697 to 9895 for women? It makes me somewhat cautious of the results—though it could just be a typo. Could the models be run using the same sample for each column?
\debug{These typos got fixed.}
\end{enumerate}

\subsubsection*{6. Quality of Communication}

\begin{enumerate}
\item There appear to be a few grammatical/typo errors in
  the paper. This is not an exhaustive list, but: 
  \begin{enumerate}
  \item pg 1 ln 51 “an” is
    not needed. 
  \item Pg 2 ln43: maybe indent this paragraph. 
  \item ln 49 “We find
    that in the case of, Estonian language is associated…” maybe missing
    the word “the”  
  \item ln 54 and 57: its instead of “it’s” 
  \item ln 58 and 59
    “that [the] Estonian…” and “boundaries, men [are] more…” 
  \item Pg. 16 ln
    45-50 The whole paragraph feels a little awkward. 
  \item I also do not know
    what “cv.ee” means.
    \debug{SKB: We should set a link or further explanations}
  \end{enumerate}
\item pg 13 ln 35. “However too little results are
  statistically significant.” Does this imply that they model is not
  useful/good? A little more explanation would be helpful.
  \debug{SR: For more explanations I reestimated Model2 for unemployment for the age groups.
I made new folder in dropbox:Tabels4\_5. Here are Estimates for the age groups for 4 sub-periods, for time period 2000-2012 and for time period 1989-2012. I also made some changes in the text.}
\debug{SKB: I did not find the tables. Where exactly did you put them? Did you include them in the text as well?}
\item pg 15 ln 60(?) “, but not precisely estimated.” More explanation on what you mean here would be helpful.
\end{enumerate}


\end{document}
